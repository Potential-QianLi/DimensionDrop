%!TEX program = xelatex

\documentclass[a4paper,oneside]{ctexbook}
\synctex=1
% oneside 使得页码都显示在右侧,使得左右是对齐的;
	% 如果真需要要打印,可将oneside变成twoside,或者既然默认如此,可直接删去oneside;

\usepackage[unicode=true,colorlinks,urlcolor=blue,linkcolor=blue,citecolor=green,bookmarksnumbered=true]{hyperref}
\usepackage{latexsym,amssymb,amsmath,amsbsy,amsopn,amstext,amsthm,amsxtra,color,multicol,multirow,bm,calc,ifpdf}
\usepackage{bookmark}
\usepackage{graphicx}
\usepackage{diagbox}
\usepackage{enumerate}
%\usepackage[numbers,authoryear]{natbib}
\usepackage{subfig}
\usepackage{listings}
\usepackage{makeidx}
\usepackage{xcolor}
\usepackage{algorithm}
\usepackage{changepage}
\usepackage{algorithmic}
\usepackage{geometry}
\usepackage{amsmath}%???
\usepackage{amsthm}%证明环境宏包
% \geometry{a4paper,scale=0.7}

\graphicspath{{figures/}}

%关于缩进
\setlength{\parindent}{0pt}
% \begin{adjustwidth}{0cm}{1cm}
% \end{adjustwidth}


% 设置目录深度之用.
\setcounter{secnumdepth}{5} % 应指允许的section number depth
\setcounter{tocdepth}{5} % 可能指非section的目录标签的深度
% QUE: 更深的目录相应的函数也可以继续写下去吗?比如多个sub的?
% 下面链接有关于目录深度的定义https://tex.stackexchange.com/questions/186981/is-there-a-subsubsubsection-command


\newcommand{\diff}{\,{\mathrm d}}
\newcommand*{\trans}{^{\mathsf{T}}}
\DeclareMathOperator*{\argmin}{arg\,min}
\DeclareMathOperator*{\argmax}{arg\,max}
\providecommand{\keywords}[1]{\textbf{Keywords:} #1}


% \renewcommand\contentsname{Contents}
% \renewcommand\refname{References}`x'
% \renewcommand\figurename{Figure}
% \renewcommand\tablename{Table}
% \renewcommand\abstractname{Abstract}

\renewcommand{\algorithmicrequire}{ \textbf{Input:}}  % Use Input in the format of Algorithm
\renewcommand{\algorithmicensure}{ \textbf{Output:}}  % Use Output in the format of Algorithm


\linespread{1.5}


\begin{document}

\title{ \Huge \textbf{树上在随机环境下的随机游走} }
	
\vspace{4em}

\author{钱力}

\date{\today}

\maketitle
\thispagestyle{empty}



%\chapter*{前言}

%\addcontentsline{toc}{chapter}{前言}

 



\cleardoublepage
\phantomsection
\addcontentsline{toc}{chapter}{目录}
\tableofcontents
\cleardoublepage

\newtheorem{thm}{定理}[chapter]
\newtheorem{que}[thm]{问题}
\newtheorem{pro}[thm]{命题}
\newtheorem{cor}[thm]{推论}
\newtheorem{defi}[thm]{定义}
\newtheorem{lem}[thm]{引理}
\newtheorem{conjec}[thm]{猜想}
\newtheorem{tri}[thm]{尝试}
\newtheorem{pf}[thm]{证明}
\newtheorem{iss}{Issue}

\chapter{排它过程的带标记粒子在树上的随机游动}

\section{规则树与GW树上的排它过程环境的带标记粒子的运动}

	


	
	\subsection{维数下降}

		\subsubsection*{基础设置}
		\addcontentsline{toc}{subsubsection}{基础设置}

			Galton-Watson树$T$:

			Define $\mathcal U = \bigcup_{k=0}^\infty \mathbb N^{k}$.
			For any  $\mathbb Z_+$-valued sequence $(l_u)_{u \in \mathcal U}$ indexed by $\mathcal U$, we say $T$ is a planer tree generated by $(l_u)_{u \in \mathcal U}$ if $T \subset \mathcal U$ and that
			\begin{align}
			T 
			= \{ u_1\dots u_m \in \mathcal U: m \geq 0, 1 \le u_j \leq l_{u_1 \dots u_{j-1}}, \forall j = 1,\dots, m\}.
			\end{align}
			We say $T$ is a \textbf{tree} if there exists a $\mathbb Z_+$-valued sequence $(l_u)_{u \in \mathcal U}$ such that $T$ is generated by it. Write $\mathbb T$ for the collection of all trees. 
			记树的顶点集为$V(T)$ as the vertices of the tree $T$,其中根结点记为空集$\phi$,边集为$E(T)$.(此句有问题,它并非是一个图.因为这里反映出根结点后代的序数,而图是没有这一点的.所以不能视作图,虽然名字可以类似地起.)
			
			其中,如果$(l_u)_{u \in \mathcal U}$i.i.d.地服从某个取值为自然数的分布$\mathbf{p}$,则其为Galton-Watson树.
			我们进一步要求,$\mathbf{p}({0})=0$,$\mathbf{p}({0})$的期望有限.

			排他随机环境$(\eta_t)_{t \ge 0}$,简记为$\eta$(包括标记粒子,因为有相互作用性)

			对于$T \in \mathbb T$, 定义一个马氏过程,其状态空间为$\{0,1\}^{V(T)}$. 称$u \in T$处有/无粒子,当且仅当$\eta(u) = 1$或$0$.
			它的生成元$\Omega$是 

			$$\Omega(f(\eta)) = \sum_{x,y \in V(T)} p(x,y) (1-\eta(x)) \eta(y) (f(\eta_{x,y}) - f(\eta)),\forall \eta \in \{0,1\}^{V(T)}$$

			其中,$p(x,y)$是一个转移概率,并且$$\forall y, \sup_y \sum_x p(x,y) < \infty$$,而$\eta_{x,y}$表示将$\eta$在$x,y$两个点的位置的状态互换一下.

			将这个随机过程称之为\textbf{排它过程},记为$(\eta_t)_{t\ge0}$.
			 
			对于初分布$\eta \in \mathbf \{0,1\}^{T}$,假设根结点恒有粒子,即$\eta(\phi) = 1$,将此粒子称之为标记粒子,其所在顶点的过程记为$(X_t)_{t \ge 0}$.那么,关于$(X_t,\eta_t)$的生成元$\widetilde{\Omega}$即...

			如果以标记粒子作为根结点,则可定义另一个过程(待定义更新树根结点的运算,),这个过程称之为环境过程,记为$(\zeta_t)_{t\ge0}$.

			由\cite{CCGS19},\cite{GS19}的结论,知道在正则树$T_d$与Galton-Waston树上,其标记粒子的轨道一定对应着一条射线,所以其可定义射线空间上的调和测度.(待定义射线空间)

			而射线空间的测度,可以诱导出树上的流(待定义流)

			待定义flow rule的定义:待参考LPP95E的第五节.

			
			
			There is no randomization at this stage. 
			We now define two levels of randomness by setting that 
			\begin{enumerate}
			\item
			$\{T; \mathbf P\}$ is a $\mathbb T$-valued random element which distributed as a Galton-Watson tree,
			\item
			and conditioned on $T$, $\{(\eta_s(u))_{s\geq 0,u \in T}; \mathbf P(\cdot| T)\}$ is an $\{0,1\}^{T}$-valued stochastic process which distributed as an exclusion process on top of $T$.

			There is no randomization at this stage. We now define two levels of randomness by
			setting that
			1. {T;P} is a T-valued random element which distributed as a Galton-Watson tree,
			2. and conditioned on T, {(η s (u)) s≥0,u∈T ;P(·|T)} is an {0,1} T -valued stochastic
			process which distributed as an exclusion process on top of T.
      

\end{enumerate}
	


	\begin{iss}[Open] ~
        \begin{itemize}
        \item[ZS:]
          LQ, please add the following changes to the paper to where you see fit:
        \item[QL:]
        	好的!我就放置于基础设置中去了.
        \end{itemize}
      \end{iss}


      \begin{iss}[Open]~
        \begin{itemize}
        \item[ZS:]
It would be helpful to put some most relevant references here in this issue.
I will read them first.
		\item[QL:]
			好的,请见下面参考文献.
        \end{itemize}
      \end{iss}

    	\subsubsection*{相关文献}
		\addcontentsline{toc}{subsubsection}{相关文献}

			已经证明维数下降的文章:

			\begin{itemize}
				\item LPP95E\cite{LPP95E}:

					不退化的GW树上的简单随机游动.
				\item LPP96\cite{LPP96B}:

					特别贡献:

					解决了死枝的情形。

					本质的证明,一如R17所见到的,从HARM到UNIF,其实都没有仔细说清楚,而是直接引用了LPP95E,重点应当是Thm7.1。

					另外证明了,如何找一列子树,使维度逼近到目标。
				\item R17\cite{R17}:
				
					UNIF的维数是log m。但是这里用到了它的维数吗,似乎也很间接?

					没有用到它的维数。

					用Shannon不等式。

					再利用UNIF的定义式,事实上开始涉及GW树极限的一些量,然后与m取得联系。
				\item LP14\cite{LP14}:

					于Chapter16给出了LPP95,96内容的重新证明.感觉有一些内容是非常相似的.但由于汇总更好,且时间更晚,所以相比于那两篇,这里更值得阅读.
			\end{itemize}

			关于树上排它环境的结论:

			\begin{itemize}
			\item CCGS19\cite{CCGS19}:

				介绍了规则树上带标记的排它过程的粒子运动,速度及CLT.

				主要使用的工具是引入了horodistance.

				可以介绍一下其与GS19的异同.

			\item GS19\cite{GS19}:

				GW树上排它标粒的速度及CLT.

			\item S70\cite{S70}:

				CCGS19中提到,To give limit laws for the position of a tagged particle is a classical problem, which was already mentioned in Spitzer’s work [11].

			\item S87\cite{S87}:

				CCGS19中提到,For translation invariant transition probabilities (which are not concentrated on the nearest neighbors in the one-dimensional case), Saada established a law of large numbers in [8].

			\item O0:

				好像只是一篇Leture,故而网上找不到资源.是介绍随机环境下的EP.

				前面的部分,似乎并不是那么得有用.
			\end{itemize}

			\qquad 

			\qquad 

			

			

			

		\subsubsection*{初步讨论}
		\addcontentsline{toc}{subsubsection}{初步讨论}

			直观(有时间将其整理成严格的语言):

				\begin{conjec}

					认为会有DD的理由(dimensional drop).

					首先,排它环境的减速效应主要理由是:带标记粒子最终会往下走;带标记粒子向下的过程中,一定会有普通粒子往上走,而这些粒子轨道对应的射线,是上穿过根结点再从别处往下的.(严格想法待参考CCGS17与GS18)

					另一点,"DD的现象的发生",直观上对应于随机运动的行为是:标记粒子的轨道最终对应的射线,将几乎在每一代成比例地不去一些射线.或者说,对于第n代顶点作为标粒射线经过的概率,会渐近成比例地有一些粒子的概率的阶小于另一批粒子的概率(比例不依赖于n).(而这些比例比较低的射线,最终随着HARM的flow rule的小概率的积累,最终使这些集合的测度为0.)(严格的想法也许在读了LPP95的第九章,或者LPP96的Coro5.3,先看看这些取HARM概率接近1的射线所形成的最小子树是什么样子;或者应该直接从维数的定义来讨论)(这事实上,就是flow rule与带射线树的平稳遍历性所想实现的结果.)

						\begin{itemize}
							\item 

							文章LPP95E的Thm9.8的确有这样的意思,直观就是:能取到HARM维数的Tn的顶点,其HARM流是在$(\exp^{(-dim-\epsilon)n},\exp^{(-dim+\epsilon)n})$这样的阶数中的,对于任意的epsilon.

							但是Thm9.9才给出了具体$T^(\epsilon)$的构造,需要进一步去看一下.(待展开)

							LPP96的Coro5.3,则是让上面的$\epsilon$带上下标n并不断趋于0.并以这种方式选出了一些顶点.
						\end{itemize}

					所以,我认为在quenched情形下,即假设环境粒子给定时,由于环境粒子始终有一些在向上"冲",所以就让标记粒子每次向下时,到了每个点,都有成比例的后代因为环境粒子的阻挡,而更难以去.

					所以quenched会有DD.

					而如果quenched有这个结论,由holder exponent来看,anealed的维数正是quenched维数关于环境的平均,所以annealed的结果必然也是如此的.
				\end{conjec}

				\begin{conjec}
					\ 

					认为没有DD的理由.

					首先明确一点:环境的随机有三种:底树的随机,初始普通粒子的随机,粒子演化的随机.所以annealed与quenched这种双重随机的词可能不适用于当前的描述了.建议用给定"树/初始粒子/时间演化"这种说法来讨论.

					上面讨论存在DD的直观所质疑的地方是:仅仅考虑了初始普通粒子的随机,没有考虑粒子演化的随机.					

					关键是时间演化的随机.给定初始粒子时,普通粒子的轨道对应射线是否上穿根结点,多大程度受初始粒子的构型所决定的,又多大程度取决于接下来的时间演化呢.

					当然,可能还是会受到初始粒子构型的影响.比如,如果一个粒子的上面几代都密布了粒子,下面几代恰好粒子较少,那么这个粒子轨道的射线更少的可能上穿根结点;但是,上面密度较高的粒子,又有更大的可能其轨道射线是上穿根结点的.但是这个例子不够促成无DD,因为:从标粒的角度看,如果有一处附近恰好普粒太多,"堵了"起来,则此处标粒的最终轨道更倾向于回避,即认为时间演化解决不了这一点.

					所以是否均匀,全决于哪些粒子会向阻挡的可能性是否分布均匀了.这与问题"到达根结点的粒子的范围",是有关的.

					这个问题等于,给定确定初始构型的排它过程.也许可以了解一些收敛域的问题.
				\end{conjec}

			动机:

				调和测度会反映出粒子更可能会走的地方.这已经在LPP95E与LPP96B中有所反映.

			\begin{que}
				quenched的调和测度的维数的关于环境的平均,与annealed的调和测度的维数,似乎不太一样.

				比如规则树下,无论是静态随机环境,还是排它的动态随机环境,它们在上述的两个量的差别都是挺大的:前者还是有环境不均匀的性质,后者则显然是均匀的.

				而处理静态随机环境的R18,显然关心的是不均匀的结论.

				问题一:是不是真得不一样;事实上,它们都是只需要计算关于HARM的那个量.

				\begin{itemize}
				\item 稍想一下就的确觉得不一样了.直接anealed 的量,先关于环境平均了,然后再log的.而后者则反过来.所以Jensen应当是能做一个大概.(待写清楚.)
				\end{itemize}

				问题二:为什么要关于quenched的调和测度再平均之后的结论呢,这个量似乎没什么意义?即使不关心,也依然可以讨论维数下降,比如,证明对于每个随机环境下都会发生?
			\end{que}

		\subsubsection*{维数的计算}
		\addcontentsline{toc}{subsubsection}{维数的计算}

			在LPP95E的第四节的套路容易在这里使用:

			定义好两个ray的度量关系(保证某个轨道的开球邻域是ray上某个点以下所有的ray),而且第n层的开球邻域是n的指数。然后Holder exponent是关于射线测度与射线的函数。如果可以算出其关于HARM测度在几乎处处地为常数,那么Holder dim也是这个常数。(LPP95的Lemma4.1)

			接下来,存在两种调和测度需要考虑:关于排它环境quenched或者annealed的两种.而参考一下R17的结果,其关于随机环境亦有q与a两种,但是只计算了a的情形.q处理起来也许更加困难,也许需要其它有效的方法.故而都将考虑进来.

			两种HARM的定义:
			$$HARM^{\eta}_T(A) := P^{\eta}_T(Ray(X)\in A)$$
			$$HARM_T(A) := P_T(Ray(X)\in A)$$

			其中,$A \subset \partial T$,$\eta$表示排它环境$(\eta_t)_{t \ge 0}$,$X$表示带标记粒子的运动$(X_t)_{t \ge 0}$,以及相应的两个测度对应着关于排它环境给定或平均的RW的运动.

			这两个测度的Holder exponent所需要计算的量:对于$a.s. - T, \forall \xi = (\xi_n)_{n \ge 1} \in \partial T$,
			\begin{align}
				\lim\limits_{n}\frac{\log HARM^{\eta}_T(\xi_n)}{n}
				\label{holderexponent}
			\end{align}


			\paragraph*{尝试一:类似flow rule的展开}
			\addcontentsline{toc}{paragraph}{尝试一:类似flow rule的展开}\quad % 以换行

				首先, 依然是希望考虑有flow rule的性质能将$HARM^{\eta}_T(\xi_n)$展开(原先模型成立的理由\ref{origin flow rule}).但是事实上展开的过程会有问题,以quenched为例:
				\begin{align*}
					HARM^{\eta}_T(\xi_n)(\xi_n) &= P^{\eta}_T( \xi_n \in Ray(X) ) \\
					&= P^{\eta}_T( \xi_n \in Ray(X) | \xi_1 \in Ray(X))  P^{\eta}_T( \xi_1 \in Ray(X) )
				\end{align*}
				
				此时可能没有理想的等式
				\begin{align}
					P^{\eta}_T( \xi_n \in Ray(X) | \xi_1 \in Ray(X)) 
					= P^{\eta(T[\xi_1])}_{T[\xi_1]}( \xi_1^{-1}\xi_n \in Ray(X) )
					\label{shiftdown}
				\end{align}
				其中右式等于$HARM^{\eta(T[\xi_1])}_{T[\xi_1]}(\xi_1^{-1}\xi_n)$.

				仿照这个证明套路的问题:

				关键的问题便在于,动态随机环境$\eta$是无法让$\xi_1$上下两侧的普通粒子的演化完全独立地进行,其相互作用很可能的一种情况是:标记粒子从$\xi_1$处离开走到下面,然后$T[\xi_1]$内外的普通粒子开始交流.

				于是首先会带来沿射线下移的困难,但是解决方案见下\ref{trial};
				另一个问题是flow rule怎么从quenched到annealed,见\ref{no indpt};

				\begin{que}[相互作用刻画:两侧树谁吸引]
					\label{flow}
					\ 

					关于相互作用一个基本的问题——那到底会哪边会吸引粒子,还是两边的粒子数保持不变?由于两边的粒子数都是无穷的,所以可以以某个边视作流的检测,看看粒子变化的情况.

					目前已经知道一个事实是:如果将标记粒子放在根结点上,普通粒子依然是满足Bernoulli分布的.那么似乎感觉两侧差不到哪里去.否则,如果会有一方向必然地流向另一方的话,其中一处的粒子概率会下降,而另一处会上升?这个说法倒也未必正确.因为存在一种可能否定的情形:如果各种都整体有这样的一种流动感的话,其实是有流,但是也依然各种均匀的.

					直观感觉,也不知道哪边会更有吸引力.也许算一下$P_x(x\in \xi)$对于不同的$x$与$\xi$,会提供一些观察的角度.

				\end{que}



				\subparagraph*{Quenched HARM}
				\addcontentsline{toc}{subparagraph}{Quenched HARM}\quad

					对于quenched情形,有如下几种尝试:

					\begin{itemize}
						\label{trial}
						\item 沿着射线下移的两种方案:\ref{AGW}:考虑AGW下的情形;\ref{phistar}:只留下一个父结点.
						\item \ref{error}:计算上面等式\ref{shiftdown}左右两边的误差,看看其在\ref{holderexponent}中,这些误差log再求和之后,是否可以在比上n之后消失.
						\item \ref{LE}:事实上,射线就是随机运动去环的结果.高一帆读过L91相关内容,里面有很多涉及格林函数的内容.当然,排它环境破坏了马氏性,所以很多东西也许得重新推导一遍.
					\end{itemize}

					\begin{tri}[考虑AGW下的情形]
						\label{AGW}
						$P^{\eta}_T( \xi_n \in Ray(X) | \xi_1 \in Ray(X))$所以对应的事件是这样:如果在标记粒子最后一次来到$\xi_1$的时刻来看,它只在$T[\xi_1]$的范围内活动;而普通粒子仍然可以在整个树上运动.(未必如此,标记粒子的转移概率可能会发生变化,普通粒子亦如此;但是可能最终的ray的分布是一样的,或者差不多的.)

						所以不妨将考虑将$\xi_1$作为根结点的新树$Move(T,\xi_1)$.如果T是满足"增强GW树(augmented GW)"的分布AGW的话,$Move(T,\xi_1)$依然服从AGW的分布.

						所以就是说,$Move(T,\xi_1)$上,新根结点有一个标记粒子,除根结点的有一棵子树之外,都可以在其它的子树下运动;普通粒子则在全树上自然运动.
						假设这样的概率是$\sim{P}$...

						那么此时,条件概率是否等这个新分布呢?条件只是说一定会到$\xi_1$并不再上去.这还是有很大的不一样的.
					\end{tri}

					\begin{tri}[非AGW的变化,而是类似R17给根结点加一个父结点/蓄水池]
						\label{phistar}
						这也许是另一种方法.

						主要思路是:当标记粒子运动根结点的后代$\xi_1$不再回去时,排它环境虽然受到$\xi_1$上面的影响,但是是否可以约化为,一个$\xi_1$上面且个蓄水池:空着时会以一定概率产生粒子并向下运动;有粒子时以一定概率将其吃掉.

						上述新行为的速率也许需要仔细思考一下.

						直觉的问题:那么,标记粒子是否可以去这个新父结点呢?如果不可以,一以贯之也可以接受;如果可以去,那么exit time需要定义成一种新的量,即可以去新父结点,但是不会走向其它新父结点的其它地方.

						具体待与赵林杰讨论.
					\end{tri}

					\begin{tri}[计算下移后的误差]
						\label{error}
						\ 

						具体问题即:
						$$P^{\eta}_T ( \xi_n \in Ray | \xi_1 \in Ray)-P^{\eta(T[\xi_1])}_{T[\xi_1]} ( \xi_1^{-1}\xi_n \in Ray )$$
						关于n的阶.

						(问题:其它的层呢,如果是想迭代下去,考虑其是连乘再log的形式,感觉不如考虑成其比之后与1的误差.)

						可行充分的条件是:只需要误差,作log,求和,比上n,会收敛到0,便有相应的结果.

						尝试解决:这该怎么考虑呢?感觉非常困难.


					\end{tri}

					\begin{tri}[去环的结论]
						\label{LE}
						先读一下Lawler的文章,比如L91,看能不能模仿吧.

						其实有点轨道划分的意思.在\ref{pathcut}中有所讨论.
					\end{tri}

					\begin{que}[带标记粒子的环境(Palm)测度并不再平稳]
						
						一种解决方案是,开始时刻直接以不变分布Bernoulli分布生成,由于排它过程的补集依然是排它的,所以无论是1还是0,都将标记下去,看其运动.

						好像是从标记粒子的角度来看,周围的环境始终是不变的.这个定理应当可以解决我的问题.

					\end{que}

					\begin{que}[之前文章中HARM的flow rule将如何证明]
						\label{origin flow rule}
						\ 

						非证明必须的内容,乃其它内容:在GW简单随机游动,$\lambda$-biased随机游动,随机环境下的$\lambda$-biased随机游动,这三个模型中,如何证明:
						$$HARM^\omega_T(\xi_2) = HARM^\omega_T(\xi_1)HARM^{\omega[\xi_1]}_{T[\xi_1]}(\xi_1^{-1}\xi_2)$$

						此式主要是讲随机环境模型下的,$\omega$表示随机环境.其它模型也是类似的.

						表示主要是想证明:
						$$P^\omega_T( \xi_2 \in Ray(X) | \xi_1 \in Ray(X) ) = P^{\omega[\xi_1]}_{T[\xi_1]}( \xi_1^{-1}\xi_2 \in Ray(X) )$$

						如果前者在给定${\xi_1 \in Ray(X)}$的条件下,仍然有随机游动是马氏的,可能这个问题就比较容易解决?

						\qquad 已有文献是怎么解决的?


						\begin{itemize}
							
							\item LPP95的11页第7行,LP14的Chapter16的第5节,都说HARM是flow rule是容易验证的;LPP96没出现过flow rule这个词.

							\item R17则有一些线索.

								\begin{itemize}
							
									\item 首先,1.5节有一句简略说明其成立.其中,强调了一下条件得是everywhere transience.这个性质也是自然的,它保证暂态从而收敛到射线.

									\item 引理4.3的证明方法或许可以借鉴.经计算(\ref{r17lem4.3}),得到了
									$$P(ij\in Ray | i \in Ray) = P(ij\in Ray | \tau_\phi = \infty)$$
									但是未完成证明.

									\item Lemma4.1是一个很特殊的小事实.但是我觉得也许它对于更一般的情形:即phi可以走到上面更远处,也依然是成立的?

								\end{itemize}

						\end{itemize}
							

						\qquad 目前的想法:

							给定尾事件,的确是马氏的.看看能不能用.

							整理之前按R17的想法.\ref{fr}

							给定tau v=无穷时,是否是马氏?
								读木师姐的证明过程;
									结果是正确的.事实上,也是很简单的证明.\ref{mu}
								也可读一下LP14的前二十页,如黄翔宇所说(由于已经解决,故未读);

							孙振尧的讨论,似乎给出了更关键的想法\ref{sun},从而完成了证明.

								(待整理:如果有一些相关但是并非证明关键的部分,也许可以扔进后面的地方.这里只给一个链接就好.)

								以及由此而来的新问题,$Z^L$与$Ztuta^Ltuta$又是什么分布?

									算了一下,将L展开,应当其仍是马氏的,但是假定了永不回来的条件之下.前面已经证明了其是马氏的.

								关于这两个分布为什么其HARM又一样的另一个直观想法:
								
								关于tau=无穷取条件的新马氏链的转移概率矩阵想象一些其形式,原转移概率乘上一个关于首尾的分式.所以对于一个轨道而言,其实其概率与原马氏链的差别,仅仅依赖于轨道的首尾两个.如果是无穷射线,而且假设足够远都是一致收敛的,那么的确是差不多的东西.

						进一步的问题:LPP95中HARM的flow rule,是对GW树成立,还是对AGW成立?猜想:应当都仍然是成立的.差别仅仅体现在是否能找到了一个关于射线下移算子的不变性了.

								
					\end{que}
	
				\subparagraph*{Anealed HARM}
				\addcontentsline{toc}{subparagraph}{Anealed HARM}\quad

					有如下几种尝试方案:

					\begin{itemize}
						\item \ref{separate},等价于强行剪枝.

						\item \ref{free tagged},从标粒作为参考中心,让标粒随意移动.

						\item \ref{pathcut},轨道划分的角度去看.
					\end{itemize}

					\begin{que}[可能缺乏树的两侧的独立性]
						\label{no indpt}
						\ 

						如果想直接依照R17证明的套路,事实上,还是存在很多的问题的.

						这里可能有一个问题:R17的flow rule.其实都是annealed下的.其证明目前还在想,猜想其证明思路是,先有quenched的结论,然后取环境的期望得到的.诚如是,则有问题.

						事实上右式是两项相乘,两项分别只由$\xi_1$的一侧来决定.但是R17好像证明了这两侧的环境是独立的,从而在关于环境平均时,可以分别变成annealed结果.
					\end{que}

					\begin{tri}[拆成两部分不相关的排它环境]
						\label{separate}
						\ 

						想到一种尝试方案:定义一个新的排它环境,让$\xi_1$两侧进行只在其内部的排它运动.好像排它有结果说对称的就一定有乘积测度作为不变分布(但是上时可能未必对称).所以是否此时从Annealed来看,即可以视为让两边独立地演化.

						那其实就意味着,标记粒子不再去的部分,可以丢掉了.

						如果这个想法是可行的,前提一定是:问题\ref{flow}会显示其流以概率1的轨道保持在0附近的振辐.但是这个感觉可能吗?

						问题:普通粒子的运动还保持不变吗?从annealed意义的确结果一样,但是从quenched意义来看,也可以接受吗?

						问题:标粒的Ray的分布没有变化吗?
					\end{tri}

					\begin{conjec}[以标粒为根结点时,标粒是否可以任意移动,都保持不变分布]
						\label{free tagged}
						\ 

						(如猜想成立,则可转成尝试)

						如题的问题.此时标粒的移动,相当于认为成是环境整体的移动.但是由于处处都是bernoulli分布,所以移动也还差不多吗?

						待参考CCGS中的证明即可知之.

						猜想
					\end{conjec}

					\begin{conjec}[刷新过程]
						\label{refresh}
						\ 

						quenched到给定普通粒子初始构型,并开始演化;每次标记粒子跳了一步,普通粒子便按bernoulli刷新一次构型,再次演化,直到标记粒子再跳了一步.

						这个过程与原过程的关系如何?如果知道这个过程的性质,能算出原过程的什么性质?

						\begin{itemize}
						\item[定义]~
						考虑其生成元的显式表达?

						如果记$(\chi_t)_{t \ge 0}$是全空间的构型,则有$Lf(\chi) =$环境移动且刷新(?该怎么描述)+其它有粒子向空位移动.

						这还是跳过程吗?如果一个指数时间之后发生的构型还有随机性,这叫什么过程呢?是不是说,这个指数时间还需要更精确的分解?但是其可达的构型每一个概率都0,所以只能考虑有限个位置的结论了吗?(或者变一变,只刷新标记粒子到了新位置后,其周围的邻居吧.这样就变成正概率的事情,而且外面的演化其实还是会在演化的小周期里有影响的.那或许可以范围大到影响的可能性极小吗?)

						所以,可行的另一种定义方式是,用停时来描述.一旦标记粒子移动了,便用另一个独立的过程接上去.不再考虑其衔接的生成元.
						\item annealed(平均初始构型与时间演化之后)测度是一样的.
						\item 猜想:仅关注标记粒子的轨道性质,可能差别也不大.因为普通粒子仅仅是阻挡之用.

						所以要不大胆证明一下其是轨道同分布的?
						因为状态空间是离散的跳过程,所以是不是只需要证其是有限维同分布即可?
						或者,不同有限维同分布与轨道同分布的关系,但是先证一下这个结论?
						比如先考虑任给一个时间?

						\end{itemize}

						这个过程好分析吗?
						\begin{itemize}
						\item 也许更大胆地根据bernoulli乘积测度,假设普通粒子的行为了.
						\end{itemize}

					\end{conjec}

					\begin{conjec}[轨道划分的角度来看]
						\label{pathcut}
						\ 

						孙振尧回到i时,轨道划分是否可能是平稳遍历的?

						问题不够清楚:所谓的平稳遍历,是对于什么过程而言的?猜想,是不是iid生成一堆从i出发的轨道,然后只取可以回来的那些部分?

						直观感觉平稳更有可能,遍历更无可能.

						因为,平稳仅仅要求环境差别不大;

						遍历则可能要求任意可能轨道(回路),都可能走出来.

					\end{conjec}

			\paragraph*{尝试二:归纳地计算}
			\addcontentsline{toc}{paragraph}{尝试二:归纳地计算}\quad % 以换行

				也许不必寄希望于flow rule.可以考虑,直接对$\displaystyle\frac{\log HARM^{\eta}_T(\xi_n)}{n}$进行归纳地计算.

				主要思路,还是类似R17的Lemma4.3,关于停时进行一些讨论与展开,但是未必将使用遍历性等性质了.

				那我可以用什么完成这些类似于LLN的事情呢?

				粗联想:LPP95E的Thm9.8中,HIT并无flow rule,但是依然可以计算其log比n的极限.可以看看其方法.不过可能是对HIT的细节考察,未必可以很好地借鉴.

	
						
			\subsubsection*{维数下降解决之后可以思考的其它问题}
			\addcontentsline{toc}{subsubsection}{维数下降解决之后可以思考的其它问题}

				主要问题:

					\begin{que}

						Td树+EP的带标粒子,是否会表现出DD现象?
					
					\end{que}

					分析:见下.					

				证成DD后的延伸问题:

					\begin{que}

						如果证明了有DD现象.对于一个给定的GW树与初始粒子环境,是否存在极小顶点子集,使得其维数恰好取到其inf呢,或者说什么样的一列子树逼近到它?
					
					\end{que}

						读了一下R17的内容.它的维度是直接用一个定理来完成计算的.所以主要是算一下类似于强大数律的量.并没有具体地给出极小顶点子集.所以确实可以做.

						但是,也许这个问题并不难.

						比较关心的是其具体的问题:比如,一些后代比较小的枝,是不是就直接将其剪掉了?这应当不会,剪掉某个点有其之后的点,这事实上是很有问题的一件事.

						但是刚扫了一眼 LPP96B 的Coro5.3,似乎就是想说存在这样的tree集,概率特别大,然后其维数收敛过去.

							但是问题在于,没有具体看证明.

							为什么不直接来个概率1?也许是想这样逼近过去吧.但最终是要说明一个概率1的子集的?要知道,其维数的定义,是先取概率1的子集,然后不断地逼近过来的.	

				平行问题:

					\begin{que}

						将Td树换成GW树,是否有类似结论?
					
					\end{que}

					分析:

						具体见下一节.但可能这两者方法上差距不大.

						直接感觉是,既然GW树上无EP都有DD现象,那么EP之后是否强化这种差别,还是实质上淡化了,都不太确定.

						感觉需要以Td的结果为参照:如果Td+EP没有DD现象,那么可以深入思考;如果连Td+EP都有DD,那就说明对比无EP的Td,EP环境的确造成了麻烦,所以GW再进一步去看一下吧.

				变初分布:

					\begin{que}

						换成别的初始分布(或者一些固定的构型),是否会有一些分布使其更不均地出现阻塞?
					
					\end{que}

					首先得保证还是暂态到无穷远处的.

					然后是环境.环境是否还会收敛到iid (EP的吸收域问题)?如果是,这可能好办.


				进一步的问题:

					\begin{que}[更多或更少的后代是否抵消或产生DD现象]

						假设有DD现象,什么增长速度的树,面对iid生成EP的带标粒子运动,最终会不再具有DD现象?这个更好玩一点.

						或者,假设没有DD现象,如果后代的个数不是指数增长,而是更小的阶,是否可能出现DD现象呢?比如说,退成了Z1,连常返都是个问题.
					
					\end{que}

					分析:

						这可能只是对EP环境而言.如果是对于 随机环境树,可能没有这个结论.

							主要是EP会有排它环境阻滞的效果.而这里关心的问题是,后代个数的阶数,是否会影响阻滞效果.

							但是首先要搞明白iid是不是初分布,以及这是否仍是不变分布,接下来的环境该怎么处理等问题.

						似乎不靠谱.

							问题在于,那种树的维度可能本身就不好定义了(注意并不是树的维度,还是边界点/射线的维度.好不好定义,要看度量怎么选吧.)虽然目前关于维度的直观,的确是:这些射线所形成的子树的平均后个数,再log.

							维度本身想表征一种不同尺度的相似,而非不同尺度地更"复杂"吧.需要找到某个量很好地描述清楚其维度变化才是.

						换角度而言,如果只看一个点向一层,可能走到的比例的话(这才是维数的本质),似乎也仍有问题。

					\begin{que}

						如果GW树有叶子,结果又将如何呢?
					
					\end{que}

						从结果/直观上来猜测:叶子只是会增加逗留的时间,会增加每个地方都去一下的可能性?所以反而觉得DD现象不一定再成立.

						从技术上来说,猜想常返性什么应当不会改变,所以这应当保证了从射线的角度出发的调和测度的存在性.而调和测度是什么样子,可能还需要重新思考一下.这又是很难的一部分.

						但也许射线本身不太好找.

						有叶子的处理手法,在LLP96的第五章见到了一些.或许可以参考一下.其Thm5.1的证明,第一句说,前面的分解,已经将问题给解决。


	\subsection{到达根结点的粒子的范围}
		
		粒子向下暂态,但是根结点又还是Bernoulli分布,说明不断有此处的粒子下去,但是又有别的粒子上来。但是这些粒子又最终往下。所以根结点反复到来的粒子,在全时刻一定是从越来越深的下面上来的。

		我很感兴趣这些被根结点染过色的粒子有什么特征。

		陈老师提供了一种新想法:树换个根结点,其实还是树.所以,并非向上走是一件很困难的事.而是从很多粒子来看,正是应该向上走过去.陈老师认为,这个范围正是线性的速度生长开来的。猜测一下,速度是多少?是否就是其LLN的速度?

	\subsection{CLT}

		GS19\cite{GS19}给出了粒子的速度.但是中心极限定理还没给出.这也是目前可以研究的另一个问题.

		孙师兄觉得,中心极限定理是比强大数律更难的一件事情.

	\subsection{证明仓库}

		\begin{pf}[木师姐的证明]
			\label{mu}
			\ 

			对于任意的P(Xn+1=xn+1|Xn=xn,...,Xi=xi,tauy=infty).i都小于n,但是可能未必完整地取了从1到n-1的值;而所有的xi都不等于y.

			按条件概率展开,分子分母的概率分别将n时刻之前的Xi完全展开.然后此时便可以让tauy变成一个新停时:n+1之后的时间首次到y(这就是为什么要把n及之前的时刻全部给定出来的原因).由马氏性,可以将n+1(分子上)或n(分母上)之前时候的信息全部拿出来,而新tau的事情则以n+1或n时刻的位置为起点.

			然后将tau的公共因子提出,剩下的一堆求和重新并在一起,作为条件概率,并用X的马氏性.

			最终结果为$p(x_n,x_{n+1})\displaystyle\frac
			{P^{x_{n+1}}(\tau_y=\infty)}
			{P^{x_{n}}(\tau_y=\infty)}$.

		\end{pf}

		\begin{pf}[R17中用于证明HARM是flow rule的尝试]
			\label{fr}
			\ 

			欲证:
			$$P^{\phi}_{T,\omega}(ij\in Ray(X)) = P^{\phi}_{T,\omega}(i\in Ray(X))  P^{\phi}_{T[i],\omega[i]}(j\in Ray(X))$$

			\textbf{基本问题}:$P^{\phi}_{T[i],\omega[i]}$该怎么定义?毕竟此时向上的概率没有了,怎么调整?

			猜想:解决方案是,定义一个$\phi_*$,作为$T[i]$的根结点的父亲,然后假设不再到它.

			其中,已经有结论是(证明见\ref{r17lem4.3}):
			$$
			P^{\phi}_{T,\omega}(i\in Ray(X)) =  
			\frac
				{P^{i}_{T,\omega}(\tau_{\phi} = \infty)p(\phi,i)}
				{\sum_{k} P^{k}_{T,\omega}(\tau_{\phi} = \infty)p(\phi,k)}
			$$
			类似地,对于$P^{\phi}_{T,\omega}(ij \in Ray(X))$需要进一步计算下去.但是事实上会发现会多一个$\tau_phi = \infty$的存在.这是有点奇怪的事情.
			
			而AGW或者补充一个点的图似乎也没看到与其有什么关系.

			另一方面,上式的结果,让我想到这似乎正是假设$\tau_\phi = \infty$为条件时的新马氏链的转移概率矩阵,这也许会对我有一些启发.(暂态保证其为事实,这也很自然.)

			对比上面\ref{mu},上式的分母可以进一步化简,由$\sum_{k} P^{k}_{T,\omega}(\tau_{\phi} = \infty)p(\phi,k)=P^{\phi}_{T,\omega}(\sigma_{\phi}=\infty)$,其中$\sigma$表示1时刻之后的首达时.

			所以类似可以有:
			\begin{align*}
				P^{\phi}_{T,\omega}(ij\in Ray(X)) 
				&=
					\frac
					{P^{i}_{T,\omega}(\tau_{\phi} = \infty,ij\in Ray(X)) p(\phi,i)}
					{P^{\phi}_{T,\omega}(\sigma_{\phi}=\infty)} \\
				&=	P^{\phi}_{T,\omega}(i\in Ray(X))
					P^{i}_{T,\omega}(ij\in Ray(X) | \tau_{\phi} = \infty) \\
				&=	P^{\phi}_{T,\omega}(i\in Ray(X))
					p(i,j)......
			\end{align*}
			某个小尝试:由R17的Lemma4.1,很有趣,但是还是不能用在这里,因为那里的$\phi_*$有点特殊.
			
			另一个想法:

			其实由$P^{i}_{T,\omega}(ij\in Ray(X) | \tau_{\phi} = \infty)$这个形式,或许应当会有一些猜想,其该怎么定义$P^{\phi}_{T[i],\omega[i]}$.

			几点要求:

			第一是$P^{i}_{T,\omega}(ij\in Ray(X) | \tau_{\phi} = \infty)$将只依赖于i及其之后的环境;

			第二是开始的$P^{\phi}_{T,\omega}(i\in Ray(X))$也应有类似的形式.而这是简单的:比如假设向上还有一个概率到新顶点$\phi_*$.但是要需要条件概率假设不要去它那里.

			那进一步需要保证的就是,上面的概率,可以方便我最终计算出当前的概率.			
				


		\end{pf}
	

		 \begin{pf}[R17 Lem4.3 的尝试计算]
		 	\label{r17lem4.3}
		 	\begin{align*}
				P^{\phi}_{T,\omega}(i\in Ray(X)) 
				&= \sum_{\mbox{k属于第一层}}  P^{\phi}_{T,\omega}(i\in Ray(X), X_1=k) \\
				&(\mbox{由其为尾事件及马氏性}) \\
					&= \sum_{k}  P^{k}_{T,\omega}(i\in Ray(X))P(\phi,k) \\
				&(\mbox{关于是否回$\phi$而展开,注意这里没有phistar}) \\
					&= \sum_{k}  P^{k}_{T,\omega}(i\in Ray(X),\tau_{\phi} < \infty)P(\phi,k)\\
					&\quad  + P^{i}_{T,\omega}(i\in Ray(X),\tau_{\phi} = \infty)P(\phi,i) \\
				&(\mbox{强马氏性}) \\
					&= \sum_{k}  P^{\phi}_{T,\omega}(i\in Ray(X)) P^{k}_{T,\omega}(\tau_{\phi} < \infty)P(\phi,k) \\
					&\quad  + P^{i}_{T,\omega}(\tau_{\phi} = \infty)P(\phi,i)
			\end{align*}
			左式减去右式第一项,再作化简便有
			\begin{align*}
				P^{\phi}_{T,\omega}(i\in Ray(X)) \sum_{k} P^{k}_{T,\omega}(\tau_{\phi} = \infty)P(\phi,k) 
					&= P^{i}_{T,\omega}(\tau_{\phi} = \infty)P(\phi,i) \\
			\end{align*}
		 \end{pf}


		 \begin{pro}[起点无关]
		 	\ 

		 	对于任意$x\notin T[i]$,有$$P^x(ij \in Ray | i \in Ray) = P^i(ij \in Ray | i \in Ray)$$.

			直观:

				其实有一种”马氏性”的感觉.但有点不像,不像之处在于条件事件所对应的在给定了某个时刻的确切位置,这个时刻不是停时.

				这个时间可以认为成,

			证明:

				对于在上面的x,有$P^x(ij \in Ray) = P^x(ij \in Ray, \sigma_i < \infty) = P^i(ij \in Ray) P^x(\sigma_i < \infty)$.

			亦有 $P^x(i \in Ray) = P^i(i \in Ray) P^x(\sigma_i < \infty)$. 上面两式相比,于是便完成了证明.

		 \end{pro}

		\subsubsection*{原模型flow rule的证明}
		\addcontentsline{toc}{subsubsection}{原模型flow rule的证明}

		 \begin{pf}[temp孙振尧证明]
		 	\label{sun}
		 	\ 

		 	孙振尧的新想法:

		 	欲证命题:

		 		$$\displaystyle\frac
		 		{P^i_T(ij \in Ray)}
		 		{P^i_T(i \in Ray)}
		 		=
		 		\tilde{P}^{\phi}_{T[i]}(j \in Ray)
		 		$$
		 		其中,$\tilde{P}^{\phi}_{T[i]}$的定义即在图$T[i]$上,继承原有转移速率,除去在i点不再向上,向下的概率归一化.
			
			联想直观:
				
				转移速率不一样的RW,但是最终的HARM可能相同.这的确非常有趣.比如一维非对称简单RW,其HARM都是一致的.

				当然,这个成立的本质,我考虑了另一种可能的思路,即其转移概率从轨道而言本身的特点.
			
			证明直观:



			具体证明:
			
				定义轨道分布Q:其从i出发,按X的转移概率运动,如果回到i,就停止,作为一个有限长轨道;如果不回来那就作为一个无穷长轨道.

				定义一列$\{Z^{(n)}\}_{n \ge 0}$,使其i.i.d.地满足Q.

				令$L := \inf\{n: Z^{(n)}\mbox{为无穷长}\}$.

				\qquad \textbf{断言}:$\{Z^{(n)}\}_{0 \le n \le L}$与从i点出发的X同分布.
				
					断言证明直观:假设其新定义的过程叫$Y$,那么便考虑$P^i(Y_{n+1}=y_{n+1}|Y_n=y_n,...)$,按条件概率写成两个概率的商,易从这个轨道得知其之前Z的信息.于是分解成了若干Z相乘并在分式中消掉.最后一个Z由其马氏性易知等于X的相应的转移速率.
					

				
				然后通过对这一列轨道的筛选,说明其等价于从i出发只向下的转移速率的RW.
				
					直观:可以定义两种停时,一种是依次判断Z是否是一开始向下的停时$\tau_k$;另一种是Z在首次向下且无穷长的停时的k记为$\tilde{L}$,即停时记为$\tau_{\tilde{L}}$.

					于是定义新的轨道$\tilde{X} := \{Z^{(\tau_k)}\}_{0 \le k \le \tilde{L}}$.

				\qquad \textbf{断言}:$\tilde{X}$即将X中从i出发,且只向下,并将向下概率归一化之后的分布.并且将其分布记为$\tilde{P}^{\phi}_{T[i]}$.

					断言证明直观:将轨道看成整体的$\tilde{X}$,由其强大数律,易知为马氏的.所以只需要证每个小轨道也是马氏的(类似上面的套路).

					而每个小轨道的证明:

					虽然Z的长度是不确定的,但是其实可以将其延拓到更远的时间:如果是有限,那么最后必然回到i这个点,不妨将其视为吸收态,于是便可以延拓到无穷长的时间.同时,新的加时间的部分马氏性是保持的,即假设"现在"时刻取到i时,未来由于平凡,必然与过去独立.现在定义是这样的Z,开始继续处理.
					$$P(Z^{(\tau_1)}_{n+1} = i_{n+1} | Z^{(\tau_1)}_n = i_n, \mbox{及更早})$$
				
					i的要求是第一个时刻得向下,接下来就是不断地近邻;$i_n$及其之前,都不是$i$.
				
					问题:k是否为L,还是说其实并不必关心,因为这是一件不能在有限时间判断的事情,所以无论是或不是,其转移速率都一样
				
					将上面的条件概率展开,于是考虑如下
					\begin{align*}
						&\ P(Z^{(\tau_1)}_{l} = i_{l}, l = 0,...,n+1) \\
						=& \sum_{m=0}^\infty P(Z^{(\tau_1)}_{l} = i_{l}, l = 0,...,n+1,\tau_1 = m) \\
						=& \sum_{m=0}^\infty P(Z^{(\tau_1)}_{l} = i_{l}, l = 0,...,n+1,\mbox{前m个Z,都是向上的,其实只由第一步表征}) 
					\end{align*}

					由Z的独立性,便可以将其放出来,最后一个Zm的事件的概率与m无关,所以可以拿出求和之外,只视为Z1.于是求和内部变变成关于不往上作条件概率的一个Z.
					$$P(Z^{(1)}_{l} = i_{l}, l = 0,...,n+1)/P(Z^{(1)}\mbox{第一步向下})$$

					但是这只是分子的情况,对于分母的情况亦之,所以新产生条件的概率还是消掉了,于是即变成
					\begin{align*}
						&\ P(Z^{(1)}_{n+1} = i_{n+1} | Z^{(1)}_n = i_n, \mbox{及更早}) \\
						=& P(Z^{(1)}_{n+1} = i_{n+1} | Z^{(1)}_n = i_n)
					\end{align*}

					那有个问题,如果n是0,i1不就只能是i的后代了.所以i1关于i后代的求和,就并不是1了.为什么会这样?

					分母,只有零时刻的信息,但是其实已经蕴含了第一步的信息(要往下).于是从i点转移速率,就是原转移速率被归一化一下.

					所以其实转移概率差不多就是想要的样子:从i点向下是被归一化的,其余的地方与X的转移速率是一致的.
				
				\qquad \textbf{问题}:$\tilde{X}$如何与目标等式联系起来?

					第一步,目标概率比,与Ztuta从无穷序列的角度来看,是有限还是无穷的比例(即重复无穷多次的Ztuta,进行统计频率)联系起来.

					第二步,然后看看这个统计频率与Ztuta诱导出的向下运动的关系.

					第一步:

						对于Ztuta序列,定义一个示性函数:向下的,还是向下且以ij向下不回的.于是便可以统计这两种轨道的个数,从而有一个极限频数的公式.

						事实上,可以先算出每个事件发生的频数,然后做个比就可以.因为分子分母都是比上n.只要其分别收敛到一个正的数,就自然可以了.而这个iid强大数律是容易得到的.

					第二步:

						从Ztuta的无穷序列中,可以定义无穷个向下轨道.所以只需要证明这些是iid同分布的,就可以了吧.强马氏性已经保证是独立的,而同分布也将是自然成立的.这无非是另一次强大数律罢了.

					
			



		 \end{pf}

			
			



\bibliographystyle{IEEEtran}
\bibliography{1} % 所有文献

\addcontentsline{toc}{chapter}{参考文献}


\end{document}
%%% Local Variables: 
%%% TeX-engine: xetex
%%% End:

